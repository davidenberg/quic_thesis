%%%%%%%%%%%%%%%%%%%%%%%%%%%%%%%%%%%%%%%%%%%%%%%%%%%%%%%%%%%%%%%%%%%%%%%%%%%%%%%%
%%%%%%%%%%%%%%%%%%%%%%%%%%%%%%%%%%%%%%%%%%%%%%%%%%%%%%%%%%%%%%%%%%%%%%%%%%%%%%%%
%%                                                                            %%
%% thesistemplate_short.tex version 4.10 (2025/06/30)                         %%
%% The LaTeX template file to be used with the aaltothesis.sty (version 4.10) %%
%% style file.                                                                %%
%% This package requires pdfx.sty v. 1.5.84 (2017/05/18) or newer.            %%
%%                                                                            %%
%% This is licensed under the terms of the MIT license below.                 %%
%%                                                                            %%
%% Written by Luis R.J. Costa.                                                %%
%% Currently developed at Teacher services, Learning Services of Aalto        %%
%% University by Luis R.J. Costa since May 2019.                              %%
%%                                                                            %%
%% Copyright 2017-2025 aaltothesis.cls by Luis R.J. Costa,                    %%
%% luis.costa@aalto.fi.                                                       %%
%% Copyright 2017-2018 Swedish translations in aaltothesis.cls by Elisabeth   %%
%% Nyberg and Henrik Wallén henrik.wallen@aalto.fi.                           %%
%% Copyright 2017-2018 Finnish documentation in the template opinnatepohja.tex%%
%% by Perttu Puska, perttu.puska@aalto.fi, and Luis R.J. Costa.               %%
%% Finnish documentation in the template opinnatepohja.tex translated from    %%
%% the English template documentation.                                        %%
%% Copyright 2025 English template thesistemplate.tex by Luis R.J. Costa,     %%
%% Maurice Forget, Henrik Wallén.                                             %%
%% Copyright 2018-2025 Swedish template kandidatarbetsbotten.tex by Henrik    %%
%% Wallen.                                                                    %%
%%                                                                            %%
%% Permission is hereby granted, free of charge, to any person obtaining a    %%
%% copy of this software and associated documentation files (the "Software"), %%
%% to deal in the Software without restriction, including without limitation  %%
%% the rights to use, copy, modify, merge, publish, distribute, sublicense,   %%
%% and/or sell copies of the Software, and to permit persons to whom the      %%
%% Software is furnished to do so, subject to the following conditions:       %%
%% The above copyright notice and this permission notice shall be included in %%
%% all copies or substantial portions of the Software.                        %%
%% THE SOFTWARE IS PROVIDED "AS IS", WITHOUT WARRANTY OF ANY KIND, EXPRESS OR %%
%% IMPLIED, INCLUDING BUT NOT LIMITED TO THE WARRANTIES OF MERCHANTABILITY,   %%
%% FITNESS FOR A PARTICULAR PURPOSE AND NONINFRINGEMENT. IN NO EVENT SHALL    %%
%% THE AUTHORS OR COPYRIGHT HOLDERS BE LIABLE FOR ANY CLAIM, DAMAGES OR OTHER %%
%% LIABILITY, WHETHER IN AN ACTION OF CONTRACT, TORT OR OTHERWISE, ARISING    %%
%% FROM, OUT OF OR IN CONNECTION WITH THE SOFTWARE OR THE USE OR OTHER        %%
%% DEALINGS IN THE SOFTWARE.                                                  %%
%%                                                                            %%
%%                                                                            %%
%%%%%%%%%%%%%%%%%%%%%%%%%%%%%%%%%%%%%%%%%%%%%%%%%%%%%%%%%%%%%%%%%%%%%%%%%%%%%%%%
%%                                                                            %%
%% A concise template in English. For more detailed instructions in the use   %%
%% of this template and LaTeX-specific example see the English and Finnish    %%
%% templates thesistemplate.tex and opinnaytepohja.tex.                       %%
%%                                                                            %%
%%%%%%%%%%%%%%%%%%%%%%%%%%%%%%%%%%%%%%%%%%%%%%%%%%%%%%%%%%%%%%%%%%%%%%%%%%%%%%%%
%%                                                                            %%
%%                                                                            %%
%% An example for writting your thesis using LaTeX                            %%
%% Original version and development work by Luis Costa, changes to the text   %% 
%% in the Finnish template by Perttu Puska.                                   %%
%% Support for Swedish added 15092014                                         %%
%% PDF/A-b support added on 15092017                                          %%
%% PDF/A-2 support added on 24042018                                          %%
%% Layout design and typesettin changed 15072021                              %%
%%                                                                            %%
%% This example consists of the files                                         %%
%%       thesistemplate.tex (version 4.10) (for text in English)              %%
%%       opinnaytepohja.tex (version 4.10) (for text in Finnish)              %%
%%       kandidatarbetsbotten.tex (version 1.20) (for text in Swedish)        %%
%%       thesistemplate_short.tex (version 4.10) (abridged for text in        %%
%%                                                English)                    %%
%%       aaltothesis.cls                                                      %%
%%       linediagram.pdf (graphics file)                                      %%
%%       curves.pdf      (graphics file)                                      %%
%%       ledspole.jpg    (graphics file)                                      %%
%%                                                                            %%
%%                                                                            %%
%% Typeset in Linux with                                                      %%
%% pdflatex: (recommended method)                                             %%
%%             $ pdflatex thesistemplate                                      %%
%%             $ pdflatex thesistemplate                                      %%
%%                                                                            %%
%%   The result is the file thesistemplate.pdf that is PDF/A compliant, if    %%
%%   you have chosen the proper \documenclass options (see comments below)    %%
%%   and your included graphics files have no problems.                       %%
%%                                                                            %%
%%                                                                            %%
%% Explanatory comments in this example begin with the characters %%, and     %%
%% changes that the user can make with the character %                        %%
%%                                                                            %%
%%%%%%%%%%%%%%%%%%%%%%%%%%%%%%%%%%%%%%%%%%%%%%%%%%%%%%%%%%%%%%%%%%%%%%%%%%%%%%%%
%%%%%%%%%%%%%%%%%%%%%%%%%%%%%%%%%%%%%%%%%%%%%%%%%%%%%%%%%%%%%%%%%%%%%%%%%%%%%%%%
%%
%%
%% USE one of the following three \documentclass set-ups:
%% * the first when using pdflatex to directly typeset your document in the
%%   chosen pdf/a format for online publishing (centred page layout),
%% * the second for one-sided printing your thesis with the layout (wide left 
%%   margin), or
%% * the third for two-sided printing.
%%
\documentclass[english, 12pt, a4paper, elec, utf8, a-1b, online]{aaltothesis}
%\documentclass[english, 12pt, a4paper, elec, utf8, a-2b, print]{aaltothesis}
%\documentclass[english, 12pt, a4paper, elec, utf8, a-2b, print, twoside]{aaltothesis}

%% Use the following options in the \documentclass macro above:
%% your school: arts, biz, chem, elec, eng, sci
%% the character encoding scheme used by your editor: utf8, latin1
%% thesis language: english, finnish, swedish
%% make an archiveable PDF/A-1b or PDF/A-2b compliant file: a-1b, a-2b
%%                    (with pdflatex, a normal pdf containing metadata is
%%                     produced without the a-*b option)
%% typset for online document or print on paper: online, print
%%        online: typeset in symmetric layout and blue hypertext for online
%%                publishing
%%        print: typeset in a symmetric layout and black hypertext for printing
%%               on paper
%%          two-side printing: twoside (default is one-sided printing)
%%               typeset in a wide margin on the binding side of the page and
%%               black hypertext. Use with print only.
%%

%% FOR USERS OF AMS PACKAGES:
%% * newtxmath used in this template loads amsmath, so
%%   you needn't load it. If you want to use options in amsmath, load it here, 
%%   before \setupthesisfonts below to pass the options to amsmath.
%% * If you want to use amsthm, load it here before \setupthesisfonts to avoid
%%   a clash with newtxmath.
%% * If using amsmath with options and you want to use amsthm, load amsthms
%%   after amsmath, as described in the amsthm documentation.
%% * Don't use amsbsym or amsfonts. The symbols [and macros] there are defined in
%%   newtxmath and so clash if used.
%\usepackage[options]{amsmath}
%\usepackage{amsthm}

%% DO NOT MOVE OR REMOVE \setupthesisfonts
\setupthesisfonts

%%
%% Add here the packges you need
%%
\usepackage{graphicx}


%% For tables that span multiple pages; used to split a paraphrasing example in
%% the appendix. If you don't need it, remove it.
\usepackage{longtable}

%% A package for generating Creative Commons copyright terms. If you don't use
%% the CC copyright terms, remove it, since otherwise undesired information may
%% be added to this document's metadata.
%\usepackage[type={CC}, modifier={by-nc-sa}, version={4.0}]{doclicense}
%% Find below three examples for typesetting the CC license notice.

%%%%%%%%%%%%%%%%%%%%%%%%%% FOR THOSE WHO USE BIBLATEX %%%%%%%%%%%%%%%%%%%%%%%%%%
%% Package to use BibLaTeX with some settings. Adjust and add BibLaTeX settings
%% to suit you needs. The bibliography printing and associated commands are
%% below between the conclusions and the appendix.
%%
\usepackage[
backend=biber,
style=numeric-comp, % citations and references are numerical (Vancouver, IEEE)
sorting=none, % cited reference is first in the bibliography followed by all 
              % references in the database references.bib
firstinits=true, % show initial of first name in bibliography
urldate=long % date is expressed as Month dd, yyyy
]{biblatex}
\addbibresource{thesisreferences.bib}
%%
%%%%%%%%%%%%%%%%%%%%%%%%%%%%%% END BIBLATEX STUFF %%%%%%%%%%%%%%%%%%%%%%%%%%%%%%

%% Edit to conform to your degree programme
%% Capitalise the words in the name of the degree programme: it's a name
\degreeprogram{Electronics and Electrical Engineering}
%%

%% Your major
%%
\major{An appropriate major}
%%

%% Choose one of the three below
%%
%\univdegree{BSc}
\univdegree{MSc}
%\univdegree{Lic}
%%

%% Your name (self explanatory...)
%%
\thesisauthor{Eddie Engineer}
%%

%% Your thesis title and possible subtitle comes here and possibly, again,
%% together with the Finnish or Swedish abstract. Do not hyphenate the title
%% (and subtitle), and avoid writing too long a title. Should LaTeX typeset a
%% long title (and/or subtitle) unsatisfactorily, you might have to force a
%% linebreak using the \\ control characters. In this case...
%% * Remember, the title should not be hyphenated!
%% * A possible 'and' in the title should not be the last word in the line; it
%%   begins the next line.
%% * Specify the title (and/or subtitle) again without the linebreak characters
%%   in the optional argument in box brackets. This is done because the title
%%   is part of the metadata in the pdf/a file, and the metadata cannot contain
%%   linebreaks.
%%
\thesistitle{Title of the thesis}
%\thesistitle[Title of the thesis]{Title of\\ the thesis}
%%
%% Either remove or leave \thesissubtitle{} empty if you don't use it
%%
\thesissubtitle{A possible subtitle}
%\thesissubtitle[Subtitle of the thesis]{Subtitle of\\ the thesis}
%\thesissubtitle{}

%%
\place{Espoo}
%%

%% The date for the bachelor's thesis is the day it is presented
%%
\date{30 June 2025}
%%

%% Thesis supervisor
%% Note the "\" character in the title after the period and before the space
%% and the following character string.
%% This is because the period is not the end of a sentence after which a
%% slightly longer space follows, but what is desired is a regular interword
%% space.
%%
\supervisor{Prof.\ Pirjo Professori}
%%

%% Advisor(s)---two at the most---of the thesis. Check with your supervisor how
%% many official advisors you can have.
%%
\advisor{Dr Alan Advisor}
\advisor{Ms Elsa Expert (MSc)}
%%

%% If you do your thesis work in a company of other institute, give the name of
%% the company or instution here. Otherwise, leave the macro empty, comment it
%% out, or remove it. This will remove this field from the abstract page.
%%
\collaborativepartner{Company or institute name (if relevant)}
%%

%% Aaltologo: syntax:
%% \uselogo{?|!|'|aalto?|aalto!|aalto'|<empty>}
%% The logo language is set to be the same as the thesis language.
%%
%\uselogo{?}
%\uselogo{!}
\uselogo{'}
%\uselogo{aalto?}
%\uselogo{aalto!}
%\uselogo{aalto'}
%\uselogo{}
%%

%%%%%%%%%%%%%%%%%%               COPYRIGHT TEXT               %%%%%%%%%%%%%%%%%%
%%%%%%%%%%%%%%%%%%%%%%%%%%%%%%%%%%%%%%%%%%%%%%%%%%%%%%%%%%%%%%%%%%%%%%%%%%%%%%%%

%% Copyright of a work is with the creator/author of the work regardless of
%% whether the copyright mark is explicitly in the work or not. You may, if you
%% wish---we encourage you to do so---publish your work under a Creative
%% Commons license (see creativecommons.org), in which case the license text
%% must be visible in the work. Write here the copyright text you want using the
%% macro \copyrighttext, which writes the text into the metadata of the pdf file
%% as well.
%%
%% Syntax:
%% \copyrigthtext{metadata text}{text visible on the page}
%%
%% CHOOSE ONE OF THE COPYRIGHT NOTICE STYLES BELOW.
%% IF USING THE CC TERMS, CHOOSE THE LICENSE YOU WANT TO USE.
%% The different CC licenses are listed at 
%% https://creativecommons.org/about/cclicenses/.
%% If you use the icons from the doclicense.sty package, add the package above
%% (\usepackage{doclicense}).
%% IMPORTANT NOTE!! Manually write the CC text in the \copyrighttext metadata
%% text field.
%%
%% NOTE: In the macros below, the text written in the metadata must have a
%% \noexpand macro before the \copyright special character. When not in pdf/a
%% mode (i.e. a-1b or a-2b are not specified in \documentclass), two \noexpands
%% are required in the metadata text to correctly render the copyright mark in
%% the pdf metadata. In pdf/a mode one \noexpand suffices.
%%
%% EXAMPLE OF PLAIN COPYRIGHT TEXT
%% The macros \copyright and \year below must be separated by the \ character 
%% (space chacter) from the text that follows. The macros in the argument of the
%% \copyrighttext macro automatically insert the year and the author's name.
%% (Note! \ThesisAuthor is an internal macro of the aaltothesis.cls class file).
%%
%\copyrighttext{Copyright \noexpand\textcopyright\ \number\year\ \ThesisAuthor}
%{Copyright \textcopyright{} \number\year{} \ThesisAuthor}
%%
%% Of course, the same text could have simply been written as
%% \copyrighttext{Copyright \noexpand\copyright\ 2018 Eddie Engineer}
%% {Copyright \copyright{} 2022 Eddie Engineer}
%%
%% EXAMPLES OF CC LICENSE: different ways to display the same license
%% 1. A simple Creative Commons license text with a link to the copyright notice:
\copyrighttext{\noexpand\textcopyright\ \number\year. This work is 
	licensed under a CC BY-NC-SA 4.0 license.}{\textcopyright{} 
	\number\year. This work is licensed under a 
	\href{https://creativecommons.org/licenses/by-nc-nd/4.0/}{CC BY-NC-SA 4.0} 
	license.}
%
%% To get the URL of the license of your choice, go to 
%% https://creativecommons.org/about/cclicenses/, click on the chosen license
%% you want to use, and copy-and-paste the URL in the macro \href above.
%%
%% 2. A short Creative Commons license text containing the respective CC icons
%% (requires the package doclicense.sty to be added in the preamble as done
%% above) and a link to the corresponding Creative Commons license webpage (see
%% the doclicense package documentation for other license icons):
%\copyrighttext{\noexpand\textcopyright\ \number\year. This work is licensed
%	under a CC BY-NC-SA 4.0 license.}{
%	\parbox{95mm}{\noindent\textcopyright\ \number\year. \doclicenseText} 
%	\hspace{1em}\parbox{35mm}{\doclicenseImage}
%}
%%
%% 3. An expanded Creative Commons license text containing the respective CC
%% icons text and as generated by the doclicense.sty package (the license is set
%% via package options in \usepackage[options]{doclicense} above; see the
%% doclicense package documentation for other license texts and icons):
%\copyrighttext{\noexpand\textcopyright\ \number\year. This work is 
%	licensed under a Creative Commons "Attribution-NonCommercial-ShareAlike 4.0 
%	International" (BY-NC-SA 4.0) license.}{\noindent\textcopyright\ \number
%	\year \ \doclicenseThis}
%%%%%%%%%%%%%%%%%%%%%%%%%%%%%%%%%%%%%%%%%%%%%%%%%%%%%%%%%%%%%%%%%%%%%%%%%%%%%%%%


%% The English abstract:
%% All the details (name, title, etc.) on the abstract page appear as specified
%% above.
%% Thesis keywords:
%% Note! The keywords are separated using the \spc macro
%%
\keywords{For keywords choose\spc concepts that are\spc central to your\spc 
thesis}
%%

%% The abstract text. This text in one paragraph is included in the metadata of
%% the pdf file as well as the abstract page. To have paragraphs in your
%% abstract rewrite it in the abstarct environment as described below.
%%
\thesisabstract{%
The abstract is a short description of the essential contents of the thesis
usually in one paragraph: what was studied and how and what were the main
findings. For a Finnish thesis, the abstract should be written in both Finnish
and English; for a Swedish thesis, in Swedish and English. The abstracts for
English theses written by Finnish or Swedish speakers should be written in
English and either in Finnish or in Swedish, depending on the student’s language
of basic education. Students educated in languages other than Finnish or Swedish
write the abstract only in English. Students may include a second or third
abstract in their native language, if they wish. 
The abstract text of this thesis is written on the readable abstract page as
well as into the pdf file's metadata via the thesisabstract macro (see the 
comment in the TeX file). Write here the text that goes into the metadata. The 
metadata cannot contain special characters, linebreak or paragraph break 
characters, so these must not be used here. If your abstract does not contain 
special characters and it does not require paragraphs, you may take advantage of
the abstracttext macro (see the comment in the TeX file below). Otherwise, the 
metadata abstract text must be identical to the text on the abstract page.
}

%% You can prevent LaTeX from writing into the xmpdata file (it contains all the 
%% metadata to be written into the pdf file) by setting the writexmpdata switch
%% to 'false'. This allows you to write the metadata in the correct format
%% directly into the file thesistemplate.xmpdata.
%\setboolean{writexmpdatafile}{false}


%% All that is printed on paper starts here
%%
\begin{document}

%% Create the coverpage
%%
\makecoverpage

%% Typeset the copyright text.
%% If you wish, you may leave out the copyright text from the human-readable
%% page of the pdf file. This may seem like a attractive idea for the printed
%% document especially if "Copyright (c) yyyy Eddie Engineer" is the only text
%% on the page. However, the recommendation is to print this copyright text.
%%
\makecopyrightpage

\clearpage
%% Note that when writing your thesis in English, place the English abstract
%% first followed by the possible Finnish or Swedish abstract.

%% Abstract text
%% All the details (name, title, etc.) on the abstract page appear as specified
%% above. Add your abstarct text with paragraphs here to have paragraphs in the
%% visible abstract page. Nonetheless, write the abstarct text without
%% paragraphs in the macro \thesismacro so that it is added to the metadata as
%% well.
%%
\begin{abstractpage}[english]
  The abstract is a short description of the essential contents of the thesis,
  usually in one paragraph: what was studied and how and what were the main
  findings.

  For a Finnish thesis, the abstract should be written in both Finnish and
  English; for a Swedish thesis, in Swedish and English. The abstracts for
  English theses written by Finnish or Swedish speakers should be written in
  English and either in Finnish or in Swedish, depending on the student’s
  language of basic education. Students educated in languages other than Finnish
  or Swedish write the abstract only in English. Students may include a second
  or third abstract in their native language, if they wish.
\end{abstractpage}

%% The text in the \thesisabstract macro is stored in the macro \abstractext, so
%% you can use the text metadata abstract directly as follows:
%%
%\begin{abstractpage}[english]
%	\abstracttext{}
%\end{abstractpage}

%% Force a new page so that the possible Finnish or Swedish abstract does not
%% begin on the same page
%%
\newpage
%%
%% Abstract in Finnish. Delete if you don't need it. 
%%
%% Respecify those fields that differ from the earlier specification or simply
%% respecify all fields.
\thesistitle{Opinnäyteen otsikko}
\thesissubtitle{Opinnäytteen mahdollinen alaotsikko}
\supervisor{Prof.\ Pirjo Professori}
\advisor{TkT Alan Advisor}
\advisor{DI Elsa Expert}
\degreeprogram{Elektroniikka ja sähkötekniikka}
\major{Sopiva pääaine}
\collaborativepartner{Yhtiön tai laitoksen nimi (tarvittaessa)}
\date{30.6.2025}
%% The keywords need not be separated by \spc now.
\keywords{Vastus, resistanssi, lämpötila}
%% Abstract text
\begin{abstractpage}[finnish]
Tiivistelmä on lyhyt kuvaus työn keskeisestä sisällöstä usein yhtenä kappaleena:
mitä tutkittiin ja miten sekä mitkä olivat tärkeimmät tulokset.

Suomenkielisen opinnäytteen tiivistelmä kirjoitetaan suomeksi ja englanniksi ja 
ruotsinkielisen vastaavasti ruotsiksi ja englanniksi. Suomen- tai
ruotsinkielisten opiskelijoiden, joiden opinnäytteen kieli on englanti, tulee 
kirjoittaa tiivistelmänsä englanniksi ja koulusivistyskielellään. Muiden kuin
koulusivistyskieleltään suomen- tai ruotsinkielisten tulee kirjoittaa
tiivistelmänsä vain englanniksi. Opiskelija voi halutessaan lisätä
opinnäytteeseensä toisen tai kolmannen tiivistelmän omalla äidinkielellään.
\end{abstractpage}

%% Force new page so that the Swedish abstract starts from a new page
\newpage

%% Swedish abstract. Delete it if you don't need it. 
%% 
%% Respecify those fields that differ from the earlier specification or simply
%% respecify all fields.
\thesistitle{Arbetets titel}
\supervisor{Prof.\ Pirjo Professori}
\advisor{TkD Alan Advisor} %
\advisor{DI Elsa Expert}
\degreeprogram{Electronik och electroteknik}
\collaborativepartner{Company or institute name in Swedish (if relevant)}
%\date{30.6.2025}
%% Abstract keywords
\keywords{Nyckelord p\aa{} svenska, temperatur}
%% Abstract text
\begin{abstractpage}[swedish]
Sammandraget är en kort beskrivning av arbetets centrala innehåll: vad 
undersöktes, hur undersöktes det och vilka var de viktigaste resultaten?

I lärdomsprov som skrivs på svenska skrivs sammandraget på svenska och engelska, 
på motsvarande sätt skrivs sammandraget på finska och engelska i lärdomsprov på 
finska. Finsk- eller svenskspråkiga studerande som skriver sitt lärdomsprov på 
engelska ska skriva sammandraget på engelska och på sitt skolutbildningsspråk. 
Studerande vars skolutbildningsspråk inte är svenska eller finska skriver 
sammandraget endast på engelska. Den studerande kan om hen så önskar lägga till 
ett andra eller tredje sammandrag på sitt eget modersmål. Sammandraget fungerar 
då ofta som mognadsprov och bör i så fall vara minst 300 ord långt. Information 
om mognadsprov på svenska finns på MyCourses:\\
\url{https://mycourses.aalto.fi/course/view.php?id=26872}.
\end{abstractpage}


\dothesispagenumbering{}

%% Preface
%%
%% This section is optional. Remove it if you do not want a preface.
\mysection{Preface}
%\mysection{Esipuhe}
I want to thank Professor Pirjo Professor and my instructors Dr Alan Advisor and
Ms Elsa Expert for their guidance.

I also want to thank my partner for keeping me sane and alive.

\vspace{5cm}
Otaniemi, 30 June 2025\\

\vspace{5mm}
{\hfill Eddie E.\ Engineer \hspace{1cm}}

%% Force a new page after the preface
%%
\newpage


%% Table of contents. 
%%
\thesistableofcontents


%% Symbols and abbreviations
\mysection{Symbols and abbreviations}

\subsection*{Symbols}

\begin{tabular}{ll}
$\mathbf{B}$  & magnetic flux density  \\
$c$              & speed of light in vacuum $\approx 3\times10^8$ [m/s]\\
$\omega_{\mathrm{D}}$    & Debye frequency \\
$\omega_{\mathrm{latt}}$ & average phonon frequency of lattice \\
$\uparrow$       & electron spin direction up\\
$\downarrow$     & electron spin direction down
\end{tabular}

\subsection*{Operators}

\begin{tabular}{ll}
$\nabla \times \mathbf{A}$              & curl of vectorin $\mathbf{A}$\\
$\displaystyle\frac{\mbox{d}}{\mbox{d} t}$ & derivative with respect to 
variable $t$\\[3mm]
$\displaystyle\frac{\partial}{\partial t}$  & partial derivative with respect 
to variable $t$ \\[3mm]
$\sum_i $                       & sum over index $i$\\
$\mathbf{A} \cdot \mathbf{B}$    & dot product of vectors $\mathbf{A}$ and 
$\mathbf{B}$
\end{tabular}

\subsection*{Abbreviations}

\begin{tabular}{ll}
AC         & alternating current \\
APLAC      & an object-oriented analog circuit simulator and design tool \\
           & (originally Analysis Program for Linear Active Circuits) \\
BCS        & Bardeen-Cooper-Schrieffer \\ %% dash between the names
DC         & direct current \\
TEM        & transverse eletromagnetic
\end{tabular}


%% \clearpage is similar to \newpage, but it also flushes the floats (figures
%% and tables).
%%
\cleardoublepage

%% Text body begins. 
%%
\section{Introduction}
\label{sec:intro}

%% Leave page number of the first page empty
%% 
\thispagestyle{empty}

This is the template file for writing your bachelor’s, master’s or licentiate 
thesis. The template contains parts and text to accommodate the different degree 
levels, bachelor’s and master’s, and so it may contain parts that may not be 
relevant to your thesis. If so, simply delete those parts in the thesis that you
do not need. This applies particularly to the abstract pages, the list of
symbols, the list of abbreviations, and appendices. You may also want to adjust
the template section titles to better suit your work.

See the file thesistemplate.tex or opinnaytepohja.tex for examples and 
descriptions on the use of various document elements like lists, figureas, 
tables equations and more. 

\subsection{Typical content in the introduction}

In principle, the introduction is like the abstract, only broader in scope, and
more detailed. The introduction generally describes the following:

%% An example of an unordered (itemised) list. Instead on the default bullet
%% point, an endash (-- in LaTeX code) is used as the item label.
\begin{itemize}
	\item[--] a description of the background of the field of study, what
	similar work others have already done, as well as an overview of the study,
	\item[--] the goals of the study,
	\item[--] the primary research question and the sub-problems in the line
	of inquiry, and
	\item[--] the scope and constraints of the study along with the main
	concepts involved.
\end{itemize}

Although the introduction is a general description of the study, be concise and 
avoid writing a lengthy introduction. A concise introduction need not have any 
subsections.

%% In a thesis, every section/chapter starts a new page, hence the \clearpage
\clearpage

\section{Literature review}
\subsection{Structure of the thesis}

The thesis comprises the front matter, the main matter and possible appendices.
The front matter in the required order is:

%% An example of an unordered (itemised) list with the default bullet point
%% label.
\begin{itemize}
	\item a cover page,
	\item a page containing copyright information,
	\item the abstract page(s),
	\item an optional preface, and
	\item a table of contents.
\end{itemize}

If the thesis contains mathematical equations, give the list of symbols used to 
represent various quantities along with the mathematical operators used.
The list must contain all the abbreviations used as well. Note that lists of
figures and tables are not required.

Note that the chapters and sections within the main matter are 
numbered, and that they appear in the table of contents. The references, or the 
bibliography, is also shown in the table of contents, but without a number 
labelling it.

The appendix or appendices, when necessary, are presented in the last part of
the thesis. They contain things like questionnaires used in the study, 
[selected parts of] data, derivations of mathematical results, a more detailed 
exposition of some aspect in the thesis, or code listings. Number them in the 
table of contents.


\clearpage

\section{Research material and methods}

This part is the core of your work, where you explain the methodological choices
you made, its limitations, how you pick your research material or subjects, the 
implementation of your study and the methods used. This section determines the 
methodological strengths and weaknesses of your thesis. Any earlier description 
of the method should limit itself to work done earlier by others. Here you tell 
your reader what you have done.

\clearpage

\section{Results}

Present the results of your study here and answer the research questions, asked 
earlier in the thesis (in the introduction, perhaps), this study strives to 
answer. The scientific value of your work is measured by the results you obtain 
along with the arguments you give to back the answers to your research 
questions.

Be critical of the significance of your results. You may critically scrutinise 
the results and your interpretation of the results here, or you may do so later 
in the chapter with the discussion of your work or in the conclusions part.

This part should discuss how reliable the data used in the study are. You may 
discuss the reliability of the conclusions drawn from the study either in this 
chapter or later in the discussions part. You may have the discussion in a 
chapter of its own, separate from the summary or conclusions.


\clearpage

\section{Summary/Conclusions}
\label{sec:summary}

This is where you tie up any loose ends. Tell your reader briefly and clearly 
what you have done, what you have discovered, and the value of your discovery 
in the context of similar work done earlier. Draw clear conclusions regarding 
the research problem, sub-problems or hypotheses. You also discuss future lines 
of study and new questions your study might have posed.

As the author of the thesis, you alone are responsible for ensuring that the 
layout, form and structure of your thesis adheres to the guidelines outlined by 
your school. This template aims to help you meet these requirements.

Finally, a dummy citation \cite{Dyson} to get a reference to an item in the bibliography.



\clearpage
%% Bibliography / list of references
%%

%%%%%% FOR THOSE WHO USE BIBLATEX %%%%%%
% Redefine 'visited on' to 'Accessed on' the 
%\DeclareFieldFormat{urldate}{%
%	(Accessed on %
%	\mkbibmonth{\thefield{urlmonth}}\addspace%
%	\thefield{urlday}\addcomma \addspace      %
%	\thefield{urlyear}\isdot)}

\nocite{*} % print uncited references in the bibliography
\printbibliography[heading=bibintoc] %, add the title to the table of contents title={References}]

%%%%%%%%%% END BIBLATEX STUFF %%%%%%%%%%

%% The hand-written bibliography
%\thesisbibliography % Required to get the bibliography title in toc and to get
                     % the page number hyperlink to the page correct.

%\begin{thebibliography}{99}
%
%  \bibitem{aaltolib} Citation Guide: Making a bibliography, \textit{Aalto 
%  	University Learning Centre}. Online article. Available  
%    \url{https://libguides.aalto.fi/c.php?g=410674&p=2797572}
%    (accessed on 14.7.2021)
%
%  \bibitem{Bringhurst} Bringhurst, R., \textit{Horizontal Motion. The Elements 
%  	of Typographic Style}, Point Roberts, WA: Hartley \& Marks, 1992. p. 26, 
%    pp.\ 25--36. Also available online as version 3.0 at  
%    %\url{https://smallpressblog.files.wordpress.com/2017/11/bringhurstelementsselections1.pdf} (accessed on 7 May 2021).
%
%  \bibitem{Dyson} Dyson, M. C., and Kipping, G. J., ``The Effects of Line Length 
%    and Method of Movement on Patterns of Reading from Screen,'' 
%    \textit{Visible Language,} vol.~2, no.~2, pp. 150--181, 1998.
%
%  \bibitem{Wikilinelength} Wikipedia contributors, ``Line length,'' 
%    \textit{Wikipedia: The Free Encyclopedia}, Wikimedia Foundation, Inc., 
%    22 July 2004.
%    \url{https://en.wikipedia.org/w/index.php?title=Line_length&oldid=997524503}
%    (accessed on 7 May 2021).
%
%\end{thebibliography}

%% Appendices
%% If you don't have appendices, your thesis ends here. Remove \clearpage,
%% \thesisappendix and the following text below. The last command of this file
%% is \end{document}.
\clearpage

\thesisappendix

\section{Contents of an appendix}
\label{app:contents}

Appendices are not essential in a thesis, and so you must plan the content of 
your thesis as if it does not contain an appendix. The appendix cannot be used 
as a dumping ground for text and ideas from an overgrown thesis.

An appendix is an independent entity, even though it complements the thesis. 
So, the appendix is not, say, just a list or image or table, but contains 
explanatory text as well that indicates the purpose of its content. It can 
contain code listings, like the one below for a simplified list of commands to 
create an appendix.

The appendix can contain figures that do not fit in to complement the text in 
the thesis. The numbering of figures is like that of equations: see figure~\ref{appfig:refraction}.

The numbering of tables is like that for equations and figures, as is evident 
from the caption of table~\ref{apptab:schedule}.

%% Example of a table in the appendix. Note how h places the table in the
%% current position.
\begin{table}[htb]
	\centering
	\caption{Caption for the table.}
	\label{apptab:schedule}
	\sffamily% change the font in the table to sans serif
	\fbox{
		\begin{tabular}{lp{0.5\linewidth}}
			9.00--9.55  & Safety instructions on the use of laboratories\\
			9.55--10.00 & Transfer to the laboratory
		\end{tabular}}
\end{table}

\end{document}
